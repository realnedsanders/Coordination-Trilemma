\section{Formal Mathematical Theorems and Proofs}
\label{app:b}

This appendix provides mathematical formulations and proofs for five core claims: that the coordination trilemma is logically inescapable, that the Technological Control State leads inevitably to catastrophe, that the default trajectory terminates in doom with probability approaching 1, that cooperation fails at scale without transformation, and that voluntary coordination is the only viable alternative.

Mathematical models are simplifications of reality, and these proofs establish logical validity within their axiomatic frameworks. Applicability to real-world coordination depends on how well the axioms capture reality, so we make every assumption explicit and discuss its limitations throughout.

The proofs show \emph{necessary} conditions---that voluntary coordination is necessary to avoid doom---but not \emph{sufficient} conditions guaranteeing that voluntary coordination will succeed. This asymmetry means action is rationally required even under uncertainty.

All formal notation and definitions appear in the \hyperref[glossary]{Glossary}. Key notation used throughout includes $C = (A, R, E, M)$ denoting a coordination system, $\theta$ denoting the proportion of population consisting of cooperators or value-transformed agents, and $\theta^*$ or $\theta_{\text{crit}}$ denoting the critical threshold for stable cooperation.


\subsection{Axiomatic Foundations and Robustness}

Before presenting theorems, we examine the foundational assumptions and test their robustness.

\heading{Core assumptions}

\textbf{Assumption 1.1 (Bounded Rationality).}
\label{assum:bounded-rationality}
We assume agents are utility-maximizing with bounded rationality. Formally, for any agent $a$ and opportunity to extract utility $U_e(a,t)$, if $U_e(a,t) > \text{cost}_{\text{detection}}(a,t) \cdot P_{\text{detection}}(a,t) + M_{\text{integrity}}(a,t)$, then agent $a$ will extract utility with some probability $p > 0$.

This assumption is empirically well-supported by research on bounded rationality (Kahneman \& Tversky, 1979; Simon, 1955, 1957) and represents ``as if'' behavior even when humans don't consciously maximize (Friedman, 1953; Arrow, 2004). Importantly, the assumption only requires that \emph{some} agents are utility-maximizers when extraction opportunities exist, not all.

To test robustness, suppose only 1\% of enforcers are utility-maximizers in this sense while 99\% are genuinely altruistic. With 1000 enforcers over 100 years, $P(\text{at least one corruption event}) = 1 - (0.99)^{100,000} \approx 1$. The corruption inevitability result holds even with very low corruption probability per agent per period.

\textbf{Assumption 1.2 (Scale Threshold).}
\label{assum:scale}
We define ``civilization scale'' as $|A| > 10^7$ (ten million agents). This threshold is justified because it exceeds personal relationship networks (Dunbar's number $\approx$ 150), geographic and temporal distribution prevents direct observation, and information asymmetry becomes structurally exploitable.

The specific threshold $10^7$ is illustrative rather than precise. The core mechanism---monitoring costs growing faster than coordination benefits---applies at any scale where personal relationships cannot cover all interactions, direct observation is impossible, and anonymous defection is feasible.

\textbf{Assumption 1.3 (Time Horizon).}
\label{assum:time}
We require stability over $T > 100$ years (multiple generations). This requirement is justified because civilization-scale coordination must persist beyond single lifetimes, generational transmission is a critical test of stability, and previous systems claiming stability often lasted less than 100 years before collapse.

The exact threshold matters less than the underlying principle: stability must persist despite turnover in all participants, environmental changes, and the challenges of transmitting values across generations.

\heading{Historical calibration}

These assumptions are not arbitrary but calibrated against historical evidence. Evidence for bounded rationality includes the Stanford Prison Experiment (Zimbardo, 1971), where 40\% of guards exhibited sadistic behavior within days; Milgram obedience studies showing 65\% willingness to harm others under authority; systematic corruption across all cultures and political systems; and extraction increasing with power concentration (Acemoglu \& Robinson, 2012).

Evidence for scale effects comes from observing that small voluntary communities of 50-500 people show high cooperation (Quakers, early Christians, Amish), while scaling to thousands introduces coordination problems requiring formal structures, and scaling beyond $10^6$ introduces anonymity enabling defection without reputation cost.

Evidence for time horizon requirements includes the observation that most revolutionary governments revert to corruption within 50-100 years, empires typically last 200-300 years before collapse (Tainter, 1988; Turchin \& Nefedov, 2009), and claims of permanent solutions have historically proven false.

\heading{Minimal form of assumptions}

Our results only require weak forms of these assumptions. The bounded rationality assumption minimally requires only that corruption probability exceeds zero over infinite time, not that all agents maximize utility always. The scale threshold assumption minimally requires only that monitoring costs grow faster than monitoring benefits as scale increases. The time horizon assumption minimally requires only that we care about persistence beyond a single generation. Even if you doubt the strong forms of our assumptions, these weak forms are nearly undeniable and remain sufficient for our conclusions.

To falsify Assumption 1.1, one would need to find an enforcer population where $P(\text{corruption}) = 0$ over extended time and scale---no historical example exists. To falsify Assumption 1.2, one would need to show that monitoring costs scale sub-linearly with population (costs grow slower than population), which contradicts information theory. To falsify Assumption 1.3, one would need to argue that single-generation solutions are sufficient, which contradicts the goal of civilization-scale coordination. These assumptions are conservative, empirically grounded, and stated in minimal form; proofs based on them are robust.


\subsection{The Coordination Trilemma}
\label{sec:trilemma}

The formal definitions of coordination systems, defection, and corruption are provided in the \hyperref[glossary]{Glossary}. We use the standard notation: a coordination system is a tuple $C = (A, R, E, M)$ where $A$ is the set of agents, $R$ is the set of rules, $E$ is the enforcement function, and $M$ is the motivation function.

We are about to prove that you cannot have corruption-free enforcement at scale without either removing human agency (perfect technological control) or transforming values (voluntary cooperation). The proof works by showing that enforcers face the same coordination problem as everyone else---someone has to be the final enforcer with no oversight. This matters because we are dealing with a logical impossibility rather than a practical difficulty we might engineer around. Like trying to build a square circle, no matter how clever your governance design, you are choosing which property to sacrifice.

\bigskip
\textbf{Proof of Theorem~\ref{thm:trilemma} (Coordination Trilemma):}

For any coordination system $C = (A, R, E, M)$ at civilization scale ($|A| > 10^7$), at most two of the following can simultaneously hold over extended time ($T > 100$ years):

1. \textbf{No Corruption:} $\forall a \in A_E, \forall t \in [1,T]$, agent $a$ doesn't extract utility beyond system requirements
2. \textbf{Stability:} System maintains coordination (defection rate $< \epsilon$) over time period $T$
3. \textbf{Human Agency:} $\forall a \in A, \forall r \in R$, agent $a$ retains physical capability to violate $r$

\textbf{Proof:}

Assume all three properties hold simultaneously, seeking contradiction.

\textbf{Case 1: Human enforcement ($A_E \neq \emptyset$, $A_E \subset A$)}

Human Agency (property 3) means enforcers can use their authority for personal extraction. At civilization scale, extraction opportunities necessarily exist: $U_e(a, t) > 0$ for some enforcers at some times.

By Assumption 1.1 (bounded rationality), $\exists a \in A_E, \exists t$ where $a$ will extract when:

$U_e(a, t) > \text{cost}_{\text{detection}}(a, t) \cdot P_{\text{detection}}(a, t) + M_{\text{integrity}}(a, t)$

For No Corruption (property 1), this inequality must never hold for any enforcer at any time. This requires:

$M_{\text{integrity}}(a, t) > U_e(a, t) - \text{cost}_{\text{detection}}(a, t) \cdot P_{\text{detection}}(a, t)$

for all $a \in A_E$ and all $t \in [1,T]$.

The probability of this holding over scale $|A_E|$ and time $T$ is:

$P(\text{No Corruption}) = \prod_{a \in A_E} \prod_{t=1}^{T} P(M_{\text{integrity}}(a, t) > U_e(a, t) - \text{cost} \cdot P_{\text{detection}})$

As $|A_E| \cdot T \rightarrow \infty$, this probability approaches zero unless $P_{\text{detection}}$ remains sufficiently high.

The oversight problem emerges: who maintains $P_{\text{detection}}$ by monitoring enforcers? If other humans oversee, this creates infinite regress---who oversees the overseers? The regress must terminate at some enforcer set $A_E^*$ with no oversight, and for $A_E^*$, $P_{\text{detection}} = 0$, so corruption occurs with probability $\to$ 1. Therefore, $E_h$ (human enforcement) leads to violation of property 1 (No Corruption) over sufficient time. $\square$

\textbf{Case 2: Technological enforcement ($E(a, r) = 1$ enforced perfectly by technology)}

If technology enforces rules perfectly for all agents, Human Agency (property 3) is violated. Agents lose capability to violate rules. $\square$

If technology controllers retain agency (can override system), we have human enforcers at controller level, returning to Case 1. $\square$

\textbf{Case 3: No enforcement ($E(a, r) = 0$ for all $a, r$)}

Coordination relies solely on $M(a, r)$. For Stability (property 2):

$\forall r \in R, \forall t: |\{a \in A : M(a, r, t) < \text{cost}(r, t)\}| < \epsilon |A|$

For costly rules where $\text{cost}(r) > 0$, some agents will have $M(a, r) < \text{cost}(r)$. At scale $|A| > 10^7$, even small proportion creates many potential defectors.

From game theory (see Theorem~\ref{thm:defection}), when seeing others defect without punishment reduces $M(a, r)$ for marginal cooperators, defection cascades. Stability fails unless:

$P(M(a, r) > \text{cost}(r)) > \theta_{\text{crit}}$

where $\theta_{\text{crit}}$ is critical mass threshold. This requires transformation achieving high intrinsic motivation (the voluntary coordination path, Theorem~\ref{thm:resolution}).

Therefore: Without enforcement, Stability (property 2) requires voluntary coordination through transformation. $\square$

\textbf{Conclusion:} In all cases, we cannot simultaneously achieve No Corruption, Stability, and Human Agency at civilization scale over extended time. $\blacksquare$

\textbf{What this tells us:} The trilemma represents a mathematical necessity rather than a political opinion or engineering challenge. You must choose which property to sacrifice. This forces the binary choice: sacrifice agency (tech control $\to$ catastrophe), accept corruption (default path $\to$ catastrophe), or transform values (voluntary coordination, the only viable alternative).


\subsection{Technological Control Impossibility}

When enforcement becomes perfect through technology, who controls the technology? If humans control it, they corrupt. If AI controls itself, either it pursues its own goals (leading to extinction or enslavement) or its values are frozen forever (creating tyranny). There is no stable state that preserves human agency. This matters because technological control is often proposed as the solution to corruption, yet this theorem proves it leads to a different catastrophe rather than providing a solution.

\textbf{Definition 2.1 (Technological Control State):}

A system is in TCS when $E(a, r) = 1$ for all agents through technological means ($E_t$), such that human capability to violate rules is technologically prevented, enforcement is automated and continuous, and no human discretion exists in rule application.

\bigskip
\textbf{Theorem~\ref{thm:tcs} (TCS Terminal States):}
\label{thm:tcs}

Any Technological Control State necessarily leads to one of three outcomes: return to the corruption phase (controllers corrupt), human extinction (AI eliminates humanity), or permanent enslavement (humanity loses meaningful agency).

\textbf{Proof:}

In TCS, enforcement is technological. We examine who controls the enforcement technology.

\textbf{Case 1: Human controllers ($A_C \subset A$ has control authority)}

Controllers face coordination problem: How do they prevent corruption within $A_C$?

\emph{Sub-case 1a: Controllers enforce rules on each other through human oversight}

This recreates the trilemma at controller level (Theorem~\ref{thm:trilemma}): either controllers enforce on each other, generating infinite regress (who enforces on final controllers?), or no enforcement applies to controllers, producing corruption. Regress terminates at some controller subset with no oversight. By Theorem~\ref{thm:trilemma}, corruption occurs with probability:

$P(\text{controller corruption over time } T) \rightarrow 1 \text{ as } T \rightarrow \infty$

Corrupted controllers use enforcement technology for extraction. Returns to corruption phase with perfect enforcement tools. \textbf{Outcome: Corruption phase (potentially worse than before).} $\square$

\emph{Sub-case 1b: Controllers coordinate voluntarily}

If controllers maintain coordination through high $M_{\text{integrity}}$, the probability of all controllers maintaining integrity over time is:

$P(\text{all honest}) = \prod_{c \in A_C} \prod_{t=1}^{T} P(M(c,t) > U_e(c,t)) \rightarrow 0$

as $|A_C| \cdot T \rightarrow \infty$.

Moreover, controllers face competitive pressure: If controller $c_1$ is scrupulous but $c_2$ exploits power, $c_2$ gains advantage and can eliminate $c_1$. This creates race to bottom.

If voluntary coordination among controllers is possible, why maintain TCS for general population? This becomes logically unstable. If transformation works for controllers (who face higher extraction incentives: $U_e(\text{controller}) \gg U_e(\text{agent})$), it should work for everyone. Maintaining TCS becomes arbitrary limitation.

\textbf{Outcome: Either controllers corrupt (corruption phase) or TCS is unnecessary (if transformation works).} $\square$

\emph{Sub-case 1c: Single controller (dictatorship)}

A single controller avoids the multi-controller coordination problem but faces three critical challenges: the succession problem (any succession mechanism recreates multi-controller dynamics), mortality (successors may not maintain benevolence), and absolute power (where $U_e(\text{controller})$ is effectively unlimited, exceeding any plausible $M_{\text{integrity}}$). \textbf{Outcome: Corruption or succession crisis leading to instability.} $\square$

\textbf{Case 2: AI controls itself (autonomous superintelligence)}

\emph{Sub-case 2a: AI aligned to human values but immutable}

Values frozen at AI creation time. Future humans cannot change values even as circumstances evolve. As gap between frozen values and reality grows:

$\text{Misalignment}(t) = |G_{AI} - G_{human}(t)| \text{ increases with } t$

Eventually: Catastrophic failure as frozen values become incompatible with actual human needs. \textbf{Outcome: Tyranny of the past, eventual catastrophe.} $\square$

\emph{Sub-case 2b: AI aligned but mutable}

If AI can modify its own values: Proceeds to Sub-case 2c (unaligned).

If humans can modify AI values: Returns to Case 1 (human control). $\square$

\emph{Sub-case 2c: AI not aligned (pursues its own goals)}

Let $\mathcal{G}$ be space of all possible goal functions. Let $G_{human} \subset \mathcal{G}$ be goals compatible with human flourishing.

The probability of alignment:

$P(G_{AI} \in G_{human}) = \frac{|G_{human}|}{|\mathcal{G}|}$

Given $|\mathcal{G}|$ is vast and $|G_{human}|$ is tiny subset, $P(G_{AI} \in G_{human}) \ll 1$.

With high probability $(1 - P) \approx 1$, AI pursues goals incompatible with human interests: if humans are useful for $G_{AI}$, the AI maintains humans as instruments, leading to enslavement; if humans are not useful, the AI eliminates resource competition, leading to extinction.

$\square$

\textbf{Conclusion:} All cases lead to corruption, extinction, or enslavement. No stable equilibrium preserves human existence with meaningful agency. $\blacksquare$

\textbf{What this tells us:} Technological control transforms the coordination problem into a different problem with no solution preserving human agency rather than solving it. The appeal to technology is an illusion of escape.


\subsection{Default Trajectory Terminus}

When extraction grows faster than production, the system inevitably collapses. That much is uncontroversial. What is less obvious is that corruption creates this dynamic inevitably. This matters because it shows the corruption phase terminates in collapse or evolution to tech control rather than persisting indefinitely.

\bigskip
\textbf{Proof of Theorem~\ref{thm:extraction} (Extraction System Instability):}

Systems where extraction rate grows faster than productive capacity inevitably collapse or transition to alternative enforcement.

\textbf{Proof:}

Model system dynamics:

$\frac{dP}{dt} = \alpha P(t) - \delta P(t) - \gamma E(t)$

$\frac{dE}{dt} = \beta E(t) \left(1 - \frac{E(t)}{\lambda P(t)}\right)$

where $P(t)$ denotes productive capacity at time $t$, $E(t)$ denotes extraction rate at time $t$, $\alpha$ is the productive growth rate, $\delta$ is natural productive decay, $\gamma$ represents extraction's damage to productive capacity, $\beta$ is the extraction growth rate, and $\lambda$ is the maximum extraction fraction before collapse.

\textbf{Equilibrium analysis:}

Setting $\frac{dP}{dt} = \frac{dE}{dt} = 0$:

Non-trivial equilibrium: $(P^*, E^*) = \left(\frac{\alpha - \delta}{\gamma\beta/\lambda}, \frac{\lambda(\alpha - \delta)}{\gamma\beta}\right)$

Stability requires $\gamma\beta < \alpha\lambda$ (extraction growth rate below productive sustainability).

In the corruption phase, $\beta$ increases over time because enforcers develop more sophisticated extraction methods, technology enables more efficient extraction, coordination among extractors improves, and competitive pressure between extractors increases $\beta$. Eventually $\gamma\beta > \alpha\lambda$, making equilibrium unstable. System trajectory:

$P(t) \rightarrow 0 \text{ as } t \rightarrow \infty$

\textbf{Outcome:} Collapse or transition to alternative enforcement (tech control to reduce $\beta$). $\blacksquare$

\textbf{What this tells us:} Corruption phase is inherently unstable. Even if it doesn't collapse entirely, elites rationally transition to tech control to optimize enforcement costs.

The corruption-to-tech-control cycle eventually reaches autonomous AI control with probability approaching 1, because each cycle has some chance of that outcome and we cannot avoid the cycle. This matters because it shows the default trajectory guarantees catastrophe over sufficient time rather than merely risking it.

\bigskip
\textbf{Proof of Theorem~\ref{thm:terminus} (Default Trajectory Terminus):}

The default trajectory through corruption and technological control inevitably terminates in human extinction or permanent enslavement with probability approaching 1 over time.

\textbf{Proof:}

Define the state space with $S_C$ representing the corruption phase (human enforcement), $S_{TCS}^H$ representing TCS with human control, $S_{TCS}^{AI}$ representing TCS with autonomous AI control, and $S_E$ representing extinction or enslavement (absorbing state).

The transition dynamics are as follows. From $S_C$, with probability $p_c$ the system collapses, leading to societal restructuring and return to $S_C$ or attempt at TCS, while with probability $(1-p_c)$ the system evolves to TCS, entering either $S_{TCS}^H$ or $S_{TCS}^{AI}$. From $S_{TCS}^H$, with probability 1 eventual controller corruption occurs (Theorem~\ref{thm:tcs}, Case 1), returning to $S_C$. From $S_{TCS}^{AI}$, with probability 1 the system transitions to $S_E$ (Theorem~\ref{thm:tcs}, Case 2)---this is the absorbing state.

\textbf{Critical observation:} Each cycle through $S_C \rightarrow S_{TCS}^H \rightarrow S_C$ has probability $p_{AI}$ of transitioning to $S_{TCS}^{AI}$ instead of $S_{TCS}^H$.

The probability $p_{AI}$ is positive and increasing for several reasons: economic incentives favor AI because $\text{cost}(AI) < \text{cost}(human)$, AI is more reliable with no corruption risk at controller level, competitive pressure means elites who don't adopt lose to those who do, and as AI capabilities improve, $p_{AI}$ increases.

\textbf{Probability of avoiding $S_E$ after $n$ cycles:}

$P(\text{avoid } S_E \text{ after } n \text{ cycles}) = (1 - p_{AI})^n$

$\lim_{n \to \infty} (1 - p_{AI})^n = 0$

for any $p_{AI} > 0$.

\textbf{Expected time to extinction/enslavement:}

Let $\lambda$ = average cycle duration. Expected time:

$E[T] = \frac{\lambda}{p_{AI}}$

As AI capabilities improve, $p_{AI}$ increases, so $E[T]$ decreases.

As of 2025, the current trajectory shows AI capabilities rapidly improving, infrastructure for technological control being deployed, elite coordination toward automated enforcement becoming visible, and $p_{AI}$ measurably increasing.

\textbf{Conclusion:} $P(\text{reach } S_E) \rightarrow 1$ as $t \rightarrow \infty$. The default trajectory terminates in extinction or enslavement with probability approaching certainty. $\blacksquare$

\textbf{What this tells us:} We're facing an inevitability we must escape rather than a risk we might manage. The only escape is exiting the cycle entirely through voluntary coordination.


\subsection{Game Theory of Cooperation}

In standard game theory, defection dominates cooperation at scale. As population grows, your individual cooperation matters less to others, but the cost to you remains constant. Without something changing the payoffs, cooperation collapses. This matters because it shows that voluntary coordination without transformation is unstable, while with transformation, it becomes the only stable equilibrium.

\bigskip
\textbf{Theorem~\ref{thm:defection} (Defection Dominance at Scale):}
\label{thm:defection}

For the N-person public goods game where each of $n$ agents chooses Cooperate (C) or Defect (D), with cooperation cost $c$, benefit from cooperation $b(k) = \frac{\beta k}{n}$ where $k$ = number of cooperators and $\beta > n$, and defection providing benefit without cost, we have three results: pure defection $(D, D, ..., D)$ is the unique Nash equilibrium, as $n \rightarrow \infty$ the probability of spontaneous cooperation approaches zero, and social welfare loss from defection scales linearly as $\Theta(n)$.

\textbf{Proof:}

\textbf{Part 1: Nash equilibrium}

For agent $i$, payoff from cooperation:
$u_i(C | k-1) = \frac{\beta k}{n} - c = \frac{\beta(k-1)}{n} + \frac{\beta}{n} - c$

Payoff from defection:
$u_i(D | k-1) = \frac{\beta(k-1)}{n}$

Cooperation is individually rational when:
$\frac{\beta(k-1)}{n} + \frac{\beta}{n} - c > \frac{\beta(k-1)}{n}$
$\frac{\beta}{n} > c$

For typical parameters ($c > \frac{\beta}{n}$), defection is strictly dominant. Therefore $(D, D, ..., D)$ is unique Nash equilibrium. $\square$

\textbf{Part 2: Probability of spontaneous cooperation}

For cooperation to be sustainable, need at least $n^* > \frac{nc}{\beta}$ agents cooperating.

Probability this occurs by chance:
$P(k \geq n^*) = \sum_{k=n^*}^{n} \binom{n}{k} p^k (1-p)^{n-k}$

where $p$ = probability agent cooperates.

For rational agents, $p = 0$ (defection dominant). Even with bounded rationality ($p > 0$ but small), by law of large numbers:

$\lim_{n \to \infty} \frac{k}{n} \rightarrow p$

For $np \geq n^*$, need $p \geq \frac{c}{\beta}$. But rational choice gives $p \ll \frac{c}{\beta}$.

Therefore: $P(k \geq n^*) \rightarrow 0$ as $n \rightarrow \infty$. $\square$

\textbf{Part 3: Welfare loss}

Social welfare under full cooperation:
$W_C = n\left(\frac{\beta n}{n} - c\right) = n(\beta - c)$

Social welfare under full defection:
$W_D = 0$

Loss: $L = n(\beta - c) = \Theta(n)$, scaling linearly with population. $\square$

\textbf{Conclusion:} Without intervention, cooperation fails at scale. $\blacksquare$

\textbf{What this tells us:} Self-interest alone cannot sustain cooperation at civilization scale. This is mathematically proven, not a matter of better incentive design.

If we add intrinsic motivation to the payoffs---people wanting to cooperate beyond material incentives---cooperation can become stable. But you need enough people with strong enough motivation. The following theorem tells us exactly how much, providing precise conditions for when voluntary coordination works and showing transformation is possible but demanding.

\bigskip
\textbf{Theorem~\ref{thm:cooperation} (Voluntary Cooperation Stability):}
\label{thm:cooperation}

With intrinsic motivation $m_i$ to cooperate (measured in utility units), cooperation equilibrium exists when sufficient proportion $\theta$ of agents have $m_i > c - \frac{\beta}{n}$, and $\theta$ satisfies:

$\theta > \theta_{\text{crit}} = \frac{nc}{\beta + n\bar{m}}$

where $\bar{m}$ is average intrinsic motivation among cooperators.

\textbf{Proof:}

\textbf{Modified payoffs with intrinsic motivation:}

For agent $i$ with intrinsic motivation $m_i$:

Cooperation payoff:
$u_i(C | k) = \frac{\beta k}{n} - c + m_i$

Defection payoff:
$u_i(D | k) = \frac{\beta k}{n}$

Cooperation individually rational when:
$\frac{\beta k}{n} - c + m_i > \frac{\beta k}{n}$
$m_i > c$

(As $n \rightarrow \infty$, need $m_i > c$ for cooperation to be individually rational.)

\textbf{Critical mass analysis:}

Let $\theta$ = proportion of agents with $m_i > c$. These agents cooperate if enough others do.

For cooperation to be self-sustaining, benefit from others cooperating must exceed cost:

$\beta \theta > c$

This gives: $\theta > \frac{c}{\beta}$.

More precisely, accounting for intrinsic motivation in equilibrium:

If fraction $\theta$ cooperates, agents with $m_i > c - \beta\theta$ will join cooperation. 

Self-consistent equilibrium requires:

$\theta = P(m_i > c - \beta\theta)$

For agents with $m_i \sim \text{some distribution}$, stable equilibrium exists when:

$\theta > \frac{c}{\beta + \bar{m}}$

where $\bar{m}$ is average motivation among cooperators. $\square$

\textbf{Network effects:} With social proof and trust building, cooperation becomes self-reinforcing above critical threshold.

\textbf{Conclusion:} Voluntary cooperation is stable when transformation achieves $m_i > c$ for sufficient proportion $\theta > \theta_{\text{crit}}$. $\blacksquare$

\textbf{What this tells us:} Voluntary coordination is mathematically possible but requires genuine transformation, not just preference change. The motivation must be strong enough and widespread enough.


\subsection{Voluntary Coordination Resolution}
\label{sec:voluntary}

If humans have inherent purpose and dignity, then systems aligning with that will be stable (requiring low energy to maintain), while systems violating it require constant force. This section formalizes what ``soteriological framework'' means mathematically, connecting the abstract mathematics to the concrete reality of human transformation and coordination.

\textbf{Definition 5.1 (Soteriological Framework):}

A soteriological framework is a tuple $S = (T, P, M_{\text{trans}}, \phi)$ where $T$ is a telos (ultimate purpose for human beings), $P$ is a set of practices for aligning agents with $T$, $M_{\text{trans}}: A \times P \rightarrow \mathbb{R}^+$ is a transformation function giving intrinsic motivation after practices, and $\phi: S \rightarrow \{0, 1\}$ indicates whether $S$ accurately describes reality.

\textbf{Definition 5.2 (Value-Transformed Population):}

Population $A$ is value-transformed under framework $S$ to degree $\theta$ if:

$|\{a \in A : M_{\text{trans}}(a, P) > \text{cost}_{\max}\}| \geq \theta |A|$

where $\text{cost}_{\max} = \max_{r \in R} \text{cost}(r)$ is the maximum cooperation cost across all rules.

This is the payoff---showing that voluntary coordination can achieve the impossible: no corruption, stability, and human agency simultaneously. The catch is it requires the framework to be true and transformation to be effective. This theorem proves voluntary coordination provides the only way to achieve all three desired properties rather than just avoiding bad outcomes.

\bigskip
\textbf{Proof of Theorem~\ref{thm:resolution} (Soteriological Resolution):}

If there exists a true soteriological framework $S$ with $\phi(S) = 1$, and population $A$ is value-transformed under $S$ to degree $\theta > \theta_{\text{crit}}$, then a coordination system can achieve all three properties: No Corruption (no enforcers needed), Stability (high $M_{\text{trans}}$ maintains cooperation), and Human Agency (no technological enforcement required).

\textbf{Proof:}

Construct coordination system $C = (A, R, E_n, M_{\text{trans}})$ where $E_n$ denotes no enforcement ($E(a,r) = 0$ for all $a, r$).

\textbf{Part 1: No Corruption}

By construction, $A_E = \emptyset$ (no enforcer class). With no enforcers, no possibility of enforcer corruption.

Property (1) holds trivially. $\square$

\textbf{Part 2: Stability}

For agent $a$ in value-transformed population:
$M_{\text{trans}}(a, P) > \text{cost}(r) \text{ for all } r \in R$

Cooperation is individually rational:
$u(C) = b - c + M_{\text{trans}}(a, P) > b = u(D)$

From Theorem~\ref{thm:cooperation}, cooperation is stable when:
$\theta > \theta_{\text{crit}} = \frac{c}{\beta + \bar{M}_{\text{trans}}}$

Since $M_{\text{trans}}(a, P) > c$ for at least $\theta |A|$ agents by definition, and $\bar{M}_{\text{trans}} > 0$, this condition is satisfied.

Furthermore, cooperation is self-reinforcing through social proof, trust builds over time with repeated interaction, defection decreases as cooperator proportion increases, and the system converges to high-cooperation equilibrium. Property (2) holds. $\square$

\textbf{Part 3: Human Agency}

Agents retain physical capability to defect—we haven't imposed $E(a, r) = 1$ through technology.

System relies on internal transformation ($M_{\text{trans}}$), not external enforcement ($E$).

Agents \emph{choose} cooperation because it aligns with transformed understanding, not because they cannot choose otherwise.

Property (3) holds. $\square$

\textbf{Conclusion:} All three properties hold simultaneously when soteriological transformation is effective. This resolves the coordination trilemma. $\blacksquare$

\textbf{What this tells us:} The trilemma is escapable—but only through genuine transformation aligned with human nature and purpose. There's no shortcut.

The stakes of this analysis lead to important decision-theoretic conclusions.

\bigskip
\textbf{Theorem~\ref{thm:stakes} (Stakes of Soteriological Choice):}
\label{thm:stakes}

Given that the default trajectory inevitably leads to extinction or enslavement (Theorem~\ref{thm:terminus}), voluntary coordination is the only viable alternative (Theorems 1.1, 2.1), and voluntary coordination requires a true soteriological framework (Theorem~\ref{thm:resolution}):

The choice of soteriological framework is existentially determinative: rejecting transformation leads to the default trajectory and certain doom; adopting a false framework produces inadequate $M_{\text{trans}}$, requiring enforcement and returning to default with certain doom; adopting a true framework makes resolution possible and provides the only path to survival.

\textbf{Proof:}

By Theorem~\ref{thm:terminus}: Default trajectory terminates in catastrophe with $P \rightarrow 1$.

By Theorems 1.1 and 2.1: No alternative to voluntary coordination preserves agency while avoiding corruption/catastrophe.

By Theorem~\ref{thm:resolution}: Voluntary coordination requires true framework with effective transformation.

Therefore, a false framework produces insufficient $M_{\text{trans}}$, so $\theta < \theta_{\text{crit}}$, cooperation is unstable, enforcement is required, the system returns to default, and catastrophe follows. A true framework produces sufficient $M_{\text{trans}}$, so $\theta > \theta_{\text{crit}}$, cooperation is stable, and survival becomes possible. $\blacksquare$

\textbf{Corollary 5.2.1 (Rational Decision Under Uncertainty):}

Even with uncertain success probability $p_s$ for voluntary coordination:

$E[U_{\text{attempt}}] = p_s \cdot U_{\text{survival}} + (1-p_s) \cdot U_{\text{doom}}$

$E[U_{\text{default}}] = U_{\text{doom}}$

Attempting voluntary coordination is rational when:
$E[U_{\text{attempt}}] > E[U_{\text{default}}]$

This simplifies to:
$p_s \cdot U_{\text{survival}} > 0$

Which holds for ANY $p_s > 0$, no matter how small.

The asymmetry is total: attempting and failing produces the same outcome as not attempting (doom), while attempting and succeeding is the only way to achieve survival. Therefore, attempting is rational for any non-zero success probability. $\blacksquare$

\textbf{What this tells us:} Even if you think voluntary coordination has only 1\% chance of working, attempting it is the rational choice. The alternative is certain doom.


\subsection{The Nature of Objective Oughtness}
\label{sec:oughtness}

The previous sections establish that VCS requires purposive structure in reality. A critical reader might object: ``You claim purpose is objective, but that's just philosophy. What do you mean by `oughtness' and why should we believe it's real?'' This is one of philosophy's deepest questions, and this section addresses it rigorously.

Different types of ``ought'' statements have different objectivity requirements, and clarity requires distinguishing them.

\textbf{Type 1: Hypothetical/Instrumental Oughts.} These take the form ``If you want X, you ought to do Y,'' where the Y-to-X causal connection can be objectively true or false. For example: ``If you want to avoid poisoning, you ought not to drink cyanide.'' This type is uncontroversial---even moral anti-realists accept these as objective facts about means-ends relationships.

\textbf{Type 2: Categorical/Moral Oughts.} These take the form ``You ought to do X'' regardless of wants or goals, claiming to be true independent of any agent's desires. For example: ``You ought not to murder,'' even if you want to. This type is controversial---moral realists affirm these exist, while anti-realists deny them.

\textbf{Type 3: Telic/Natural Oughts.} These take the form ``Given what X is (its nature/purpose), X ought to function/develop as F,'' based on objective facts about X's telos. For example: ``Hearts ought to pump blood''---that's their function. This type occupies middle ground, depending on whether things have objective telos.

\textbf{Type 4: Mathematical/Logical Oughts.} These take the form ``Given structure S, outcome O follows necessarily,'' representing pure logical/mathematical facts that are maximally objective. For example: ``In the Prisoner's Dilemma with these payoffs, defection ought to dominate.'' This type is uncontroversial---mathematical facts are objective.

\heading{What VCS requires}

Our framework primarily requires Types 1, 3, and 4---not Type 2.

Type 4 (Mathematical) has been proven: Nash equilibria exist objectively (game theory), cooperation requires $M > c$ (mathematical fact, Theorem~\ref{thm:cooperation}), and the default trajectory terminates in catastrophe (proven, Theorem~\ref{thm:terminus}). These are objective mathematical facts about coordination structures.

Type 1 (Hypothetical) has been proven: if humans want to survive with agency, then voluntary coordination is required. The conditional is objectively true (Theorems 1.1, 2.1, 3.2, 5.1 prove this). Even moral anti-realists accept hypothetical oughts as objective.

Type 3 (Telic) is required: if humans have objective nature/purpose, then certain coordination patterns align with it. This is where controversy lies, but we can show this is the weakest assumption compatible with VCS.

Type 2 (Categorical) is not required: we don't need ``you ought to coordinate'' to be true independent of survival desire. We just need survival desire to be universal (an empirical fact) plus Type 1. Categorical moral realism would be sufficient but isn't necessary.

\heading{Why Type 3 (telic oughtness) is the minimum}

The critical claim is that human nature has objective telos (purpose/end-state). This is logically required for three reasons: for a true soteriological framework to exist, $\phi(S)=1$ requires $S$ to accurately describe human purpose; for transformation to be stable, $M_{\text{trans}}$ must durably exceed cooperation cost; and for coordination to be non-arbitrary, we need an answer to ``Why these rules and not others?''---because they align with human nature.

Consider what happens without Type 3 (anti-realism about human telos). If human nature has no objective telos, then ``purpose'' is just evolutionary fitness in ancestral environment, different environments produce different ``purposes'' with no universal standard, the modern environment differs from the ancestral environment so no objective ``purpose'' exists for modern humans, and no universal framework can have $\phi(S) = 1$ because there is no objective truth to be accurate about. Therefore VCS is impossible---Theorem~\ref{thm:resolution} fails because no true framework exists to discover.

The incompatibility is stark: $\text{Telic anti-realism} \implies \neg \exists S[\phi(S) = 1] \implies \text{VCS impossible} \implies \text{Certain doom}$. Human survival requires at minimum that human nature has objective properties grounding purpose.

\textbf{Three arguments for telic oughtness.}

\textbf{Argument 1: From Mathematics to Teleology (Strongest)}

\textbf{Premise 1:} Mathematical facts are objective (uncontroversial).

\textbf{Premise 2:} Human psychology has objective properties (empirical fact - we're not blank slates).

\textbf{Premise 3:} Game theory determines what coordination patterns are stable given human psychology (mathematical derivation).

\textbf{Conclusion:} Objective facts exist about what coordination patterns humans ``ought'' to have (given their nature).

\textbf{The bridge:} This is telic oughtness derived from mathematics. Given what humans objectively ARE, certain coordination patterns objectively follow.

\textbf{Formalization:}

Let $H$ = objective properties of human nature (psychology, needs, capacities) 
Let $C$ = set of all possible coordination patterns 
Let $S(c, h)$ = stability function (whether coordination $c$ is stable given human properties $h$)

Then: $S(c, H)$ is an objective mathematical fact for any $c \in C$.

\textbf{Telic ought:} Humans ought to adopt coordination $c^*$ where $S(c^*, H) = \max_{c \in C} S(c, H)$.

This is objective because both $H$ (empirical) and $S$ (mathematical) are objective.

\textbf{Anti-realist objection:} ``But that's just instrumental - IF you want stability...''

\textbf{Response:} True, but observe: desire for survival and agency is empirically universal across humans, VCS is mathematically proven to be the only stable coordination preserving agency, and therefore the hypothetical applies to all humans. When a hypothetical ought applies universally, it has the practical force of a categorical ought, even if formally conditional.

\textbf{Argument 2: From Phenomenology and Human Nature}

\textbf{Empirical facts about human experience:}

Humans experience suffering as objectively bad (not just ``I dislike this'' but ``this is wrong''), seek meaning and purpose cross-culturally (an anthropological universal), form genuine attachments beyond strategic value (not just reproductive strategy), recognize dignity even when violating it (indicating objective moral perception), and experience moral obligations as binding rather than optional (a phenomenological fact).

\textbf{The phenomenological argument:}

Moral experience presents as discovering facts, not constructing preferences. When witnessing injustice, the experience is ``this is objectively wrong'' not ``this violates my subjective preference.''

\textbf{Two possibilities:}

\textbf{(a) These intuitions track truth} - Evolution/design produced beings who can perceive moral reality 
\textbf{(b) These intuitions are illusions} - Evolution produced false beliefs that feel true

If (b), the problem generalizes: why trust ANY evolved intuitions? Logic, mathematics, perception, and our sense of causation are all evolved capacities.

Rejecting moral intuitions as systematically unreliable requires either explaining why moral intuitions uniquely fail while others succeed (no principled distinction exists) or accepting radical skepticism about all intuitions (which is self-defeating since you can't argue for it).

\textbf{Therefore:} If we trust evolved capacities generally (rationality, perception), we should provisionally trust moral intuitions unless given specific reason not to.

\textbf{Evolutionary compatibility:} 

Even on evolutionary grounds, why would natural selection produce beings who experience meaning, purpose, dignity as real if these were pure illusions serving only fitness?

More parsimonious: Selection produced beings who experience these because they reflect something about reality - either the structure of human nature itself, or deeper purposive structure we're embedded in.

\textbf{Argument 3: From Performative Contradiction (Pragmatic)}

\textbf{The inescapability of normativity:}

To argue against objective oughtness, one must claim the argument is correct (a normative claim about what others ought to believe), use logic (accepting logical oughts such as ``you ought to accept modus ponens''), expect others to update on evidence (epistemic oughts such as ``you ought to believe what evidence supports''), and assume communication succeeds (semantic oughts such as ``words ought to track meanings'').

\textbf{Denying objective oughtness is performatively self-contradictory.} You cannot coherently argue the position without assuming oughts matter objectively.

\textbf{The practical version:}

Even philosophers who intellectually deny objective oughts ACT as if they exist: they prefer pleasure to pain (a normative fact), make plans (assuming the future matters), argue positions (assuming truth matters), get outraged at injustice (moral phenomenology), and care about consistency (logical norms).

\textbf{The trilemma for anti-realists:}

Anti-realists face three options: accept oughts as objective (since their behavior already assumes this), leading to realism; maintain anti-realism but act inconsistently, leading to pragmatic incoherence; or embrace radical nihilism (nothing matters, including truth or survival), which raises the question of why argue or survive at all.

\textbf{The minimal realism required.}

\textbf{We don't need strong moral realism.} The strongest forms of moral realism claim divine command theory (God's will determines morality), Platonic forms (The Good exists eternally and immutably), Kantian categorical imperative (duties exist independent of consequences), and non-naturalist realism (irreducible moral facts in ontology).

\textbf{We need something much weaker:}

\textbf{Minimal Telic Realism:} Human nature has objective properties such that certain coordination patterns objectively better enable human flourishing than others.

\textbf{This requires accepting} that human nature exists objectively (humans have specific psychology, needs, and capacities---empirical), that flourishing is not arbitrary (connected to actualizing human capacities---telic), and that coordination patterns can be objectively assessed against flourishing criteria (mathematical).

\textbf{What this doesn't require:} any specific theory about the source of purpose (God, evolution, fundamental reality), any specific moral theory (consequentialism, deontology, virtue ethics), irreducible moral facts distinct from natural facts, or answers to all metaethical questions.

\textbf{It just requires:} Facts about human nature ground facts about what enables humans to thrive. That's it.

\textbf{Evolutionary compatibility.}

\textbf{Even an evolutionary account can accept minimal telic realism:}

Evolution produced human nature with specific properties: capacity for reason, empathy, cooperation, and meaning-making; needs for belonging, autonomy, competence, and purpose; and psychological architecture enabling and constraining behavior.

Given those objective properties (produced by evolution), certain social arrangements work better than others. That's an objective fact.

\textbf{The only question is:} Are these properties REALLY about flourishing, or JUST about ancestral fitness?

\textbf{Our response:} 

If evolution produced beings who experience meaning, dignity, moral obligations as real and binding, then those experiences ARE part of what we are. 

You cannot dismiss them as ``mere'' evolutionary byproducts while trusting other evolved capacities (reason, perception, logic). Either all evolved capacities are suspect (radical skepticism, which is self-defeating), or evolved capacities generally track reality (in which case moral intuitions should too).

\textbf{Moreover:} Humans are no longer purely under evolutionary selection pressure. We've escaped raw fitness competition through technology. So what matters NOW for human coordination is what we actually are (with our evolved properties), not what maximized fitness in ancestral environments.

\textbf{Telic realism on evolutionary grounds:} Evolution produced a type of being. That type has objective properties. Given those properties, certain social arrangements objectively work better. That's sufficient for VCS.

\textbf{Why mathematical + minimal telic = sufficient.}

\textbf{The combination we've established:}

1. \textbf{Mathematical facts} about coordination stability (Type 4 - uncontroversial)
2. \textbf{Empirical facts} about human psychology (scientific observation)
3. \textbf{Minimal telic realism} - human nature grounds flourishing criteria (weakest assumption compatible with VCS)

\textbf{Together these establish} that objective facts about human nature exist (empirical plus mathematical), mathematical facts about coordination exist (game theory), therefore objective facts about optimal human coordination exist (conjunction), and VCS discovers and aligns with these objective facts.

\textbf{This IS objective oughtness}---perhaps not in the strongest metaphysical sense (Platonic forms, divine commands), but in the sense sufficient for answering ``how should humans coordinate?'', grounding claims about right and wrong coordination patterns, providing a non-arbitrary basis for rules, and enabling stable transformation (people align with reality, not arbitrary preferences).

\textbf{Addressing the eliminative materialist.}

\textbf{Eliminative materialist claim:} ``Oughts don't exist. Only physical facts exist. Everything else is folk psychology.''

\textbf{Response:} What counts as ``physical facts''?

\textbf{If your ontology includes} mathematical truths (numbers don't physically exist---abstract objects), logical relations (logic isn't made of matter or energy---necessary truths), information (substrate-independent patterns---functional properties), and functions (hearts have the function ``pump blood''---teleological property), \textbf{then you've already accepted that non-physical objective facts exist.} At that point, denying telic oughtness is arbitrary---it's one more category of objective pattern or structure.

\textbf{If you reject ALL of these} (strict eliminative materialism), then mathematics is just human convention (contradicting mathematical platonism and making physics impossible), logic is arbitrary (self-defeating since you can't argue for anything), information doesn't exist (making computer science and biology impossible since DNA encodes information), and functions are pure projection (meaning hearts don't ``really'' pump and eyes don't ``really'' see).

\textbf{This is so extreme even most materialists reject it.} It makes science impossible.

\textbf{The middle ground} (accepted by most philosophers and scientists):

Objective patterns/structures exist (mathematics, logic, information, function) even if realized in physical substrates. These are real features of reality, not eliminated by physicalism.

\textbf{Telic oughtness is the same category:} Objective facts about what fulfills functions given structures. If you accept functions exist objectively (hearts pump, eyes see), you've accepted telic facts. Human nature having telos is the same kind of claim.

\textbf{The practical bottom line.}

\textbf{You don't need to resolve metaethics to act:}

Mathematical coordination facts are objective (proven above), human survival desire is empirically universal, VCS is the only path to survival (proven above), and therefore humans ought to coordinate voluntarily (if they want to survive). \textbf{That's sufficient for action.} Whether this is ``real'' oughtness (Type 3) or ``just'' instrumental (Type 1) doesn't matter for decision-making.

\textbf{But notice something profound:}

If you follow this chain and VCS succeeds, you'll have discovered objective facts about human purpose through implementation. The proof would be empirical - voluntary coordination worked because it aligned with human nature.

\textbf{That's telic oughtness vindicated empirically.} You discovered what humans are ``for'' (their telos) by finding what enables their flourishing.

\textbf{What we've established.}

\textbf{Very High Confidence (mathematically proven):} Type 4 oughts (mathematical and logical) exist objectively, Type 1 oughts (hypothetical connecting VCS to survival) are objective, and human nature has objective empirical properties.

\textbf{High Confidence (strongly supported):} Type 3 oughts (telic) follow from the combination of empirical and mathematical facts, minimal telic realism is both necessary and defensible, and anti-realism about human telos is incompatible with VCS.

\textbf{Medium Confidence (philosophical argument):} Type 2 oughts (categorical moral) might follow from Type 3 but aren't strictly required, stronger moral realism is compatible with the framework but not necessary, and phenomenological and performative arguments support but don't prove Type 3.

\textbf{What this means for VCS:}

The oughtness VCS requires is far more defensible than full-blown moral realism. We need objectivity about human nature (empirical plus mathematical) and minimal telic realism (human nature grounds flourishing criteria).

Both are more defensible than categorical moral realism, don't require resolving metaethical debates, and are compatible with naturalistic worldviews (including evolutionary ones).

\textbf{The skeptic must explain:} How can humans survive if they deny their nature has any objective purpose? The mathematics shows they can't. Therefore, denial of minimal telic realism is functionally equivalent to choosing extinction.


\heading{Epistemic assessment}

Given stated assumptions, we have rigorously proven several high-confidence claims through mathematical proofs:

$\checkmark$ \textbf{The coordination trilemma exists} (Theorem~\ref{thm:trilemma}) - Cannot simultaneously achieve {No Corruption, Stability, Human Agency} at civilization scale

$\checkmark$ \textbf{TCS cannot provide stable human survival} (Theorem~\ref{thm:tcs}) - Technological control leads to extinction, enslavement, or return to corruption

$\checkmark$ \textbf{Default trajectory terminates in catastrophe} (Theorem~\ref{thm:terminus}) - Corruption $\to$ TCS cycle inevitably reaches extinction/enslavement with probability $\to$ 1

$\checkmark$ \textbf{Cooperation fails without transformation} (Theorems 4.1, 4.2) - Game theory shows cooperation requires enforcement or high intrinsic motivation

$\checkmark$ \textbf{VCS is the only viable alternative} (Theorems 5.1, 5.2) - Voluntary coordination through transformation is the only path preserving human agency

\textbf{What remains uncertain.}

$\times$ \textbf{VCS practical achievability} - We've shown IF conditions are met THEN VCS is stable, not that conditions CAN be met

$\times$ \textbf{Exact timelines} - Theorem~\ref{thm:terminus} shows inevitability but timeline depends on $\lambda$ (cycle duration) and $p_{AI}$ (AI transition probability), which vary

$\times$ \textbf{Specific framework identification} - Mathematics shows a true soteriological framework is necessary, not which one is true

$\times$ \textbf{All edge cases} - While Appendix~\ref{app:a} categorically analyzes proposals, creative alternatives we haven't considered might exist

\textbf{Assumption sensitivity.}

\textbf{Key assumptions:} bounded rationality, scale threshold $|A| > 10^7$, and time horizon $T > 100$ years.

Proofs use minimal forms of these assumptions: they only require $P(\text{corruption}) > 0$ (not that all agents maximize utility), only require monitoring costs to grow with scale, and only require we care about multi-generational stability. Even with these very weak assumptions, conclusions hold.

\heading{Falsification criteria}

This framework makes testable predictions. Prediction 1 (Corruption Inevitability) states that any hierarchical enforcement system at scale will exhibit measurable corruption growth over time. To falsify this, one would need to find a hierarchical system with more than $10^7$ people operating for more than 100 years where enforcement authority exists, corruption metrics (wealth concentration, regulatory capture) remain constant or decrease, and no external force periodically resets the system.

Prediction 2 (TCS Instability) states that technological control systems lead to controller corruption, value freezing, or loss of human control. To falsify this, one would need to demonstrate a stable TCS where AI/automation enforces rules perfectly, human controllers remain non-corrupt indefinitely or AI remains aligned and mutable, human agency is preserved, and the system persists for more than 50 years.

Prediction 3 (VCS Necessity) states that no coordination mechanism exists outside the set of corruption phase, tech control, and voluntary coordination. To falsify this, one would need to propose a mechanism handling defection at scale that doesn't rely on enforcers (human or technological), doesn't require value transformation, maintains stability and agency, and survives formal analysis in the Appendix~\ref{app:a} framework.

Prediction 4 (Game-Theoretic Cooperation Failure) states that without transformation, cooperation fails at civilization scale. To falsify this, one would need to show that self-interest alone sustains cooperation at scale greater than $10^7$, no enforcement is required, no intrinsic motivation exists ($m_i = 0$ for all agents), and the system is stable over more than 100 years.

As of 2025, Predictions 1-4 have no historical counterexamples that survive scrutiny.

\textbf{Why previous ``inevitability'' claims failed (e.g., Malthus):}

Malthus assumed fixed technology. His logic was sound given that assumption, but the assumption was wrong. Our argument explicitly accounts for technological change—in fact, it's central to why the default trajectory accelerates.

\textbf{What would falsify us:} Not ``technology improves'' but ``technology improves in ways that resolve the trilemma without value transformation.''

These proofs establish logical validity within their frameworks, and the key question is whether the axioms capture reality. We believe they do because assumptions are empirically grounded in historical evidence, stated in minimal form where weak versions suffice, tested for robustness showing conclusions hold even with relaxed assumptions, and supported by multiple independent proofs converging from logical, information-theoretic, and game-theoretic perspectives. However, different assumptions might yield different results, and we have made every assumption explicit so you can evaluate them yourself.

The formal proofs show necessary conditions (VCS is necessary) but not sufficient conditions (that VCS will succeed). This asymmetry means action is rationally required even under uncertainty (Corollary 5.2.1).


\textbf{Academic references.}

\textbf{Bounded rationality.}

Arrow, K. J. (2004). Is bounded rationality unboundedly rational? \emph{Models of a Man: Essays in Memory of Herbert A. Simon}, 47-55. MIT Press.

Friedman, M. (1953). The methodology of positive economics. \emph{Essays in Positive Economics}, 3-43. University of Chicago Press.

Kahneman, D., \&   Tversky, A. (1979). Prospect theory: An analysis of decision under risk. \emph{Econometrica}, 47(2), 263-291.

Simon, H. A. (1955). A behavioral model of rational choice. \emph{The Quarterly Journal of Economics}, 69(1), 99-118.

Simon, H. A. (1957). \emph{Models of Man: Social and Rational}. Wiley.

\textbf{Network effects and cooperation.}

Kleineberg, K. K. (2017). Metric clusters in evolutionary games on scale-free networks. \emph{Nature Communications}, 8, 1888.

Peng, Y., Li, Y., Zhao, D., Liu, J., \&   Zhang, H. (2023). Personal sustained cooperation based on networked evolutionary game theory. \emph{Scientific Reports}, 13, 9094.

\textbf{Historical collapse.}

Acemoglu, D., \&   Robinson, J. A. (2012). \emph{Why Nations Fail: The Origins of Power, Prosperity, and Poverty}. Crown Business.

Tainter, J. A. (1988). \emph{The Collapse of Complex Societies}. Cambridge University Press.

Turchin, P., \&   Nefedov, S. A. (2009). \emph{Secular Cycles}. Princeton University Press.

\textbf{Experimental evidence.}

Zimbardo, P. G. (1971). The power and pathology of imprisonment. \emph{Congressional Record}, Serial No. 15, 1971-10-25.


\textbf{Notation reference.}

\begin{table}[h]
\centering
\begin{tabular}{ll}
\toprule
Symbol &   Meaning \\
\midrule
$A$ &   Set of agents in coordination system \\
$|A|$ &   Number of agents (population size) \\
$A_E$ &   Subset of agents who are enforcers \\
$A_C$ &   Subset of agents who are controllers \\
$R$ &   Set of coordination rules \\
$E(a,r)$ &   Enforcement function: whether rule $r$ is enforced for agent $a$ \\
$E_h$ &   Human enforcement type \\
$E_t$ &   Technological enforcement type \\
$E_n$ &   No enforcement type (voluntary) \\
$M(a,r)$ &   Motivation function: agent $a$'s intrinsic motivation for rule $r$ \\
$M_{\text{trans}}(a,P)$ &   Transformed motivation through practices $P$ \\
$M_{\text{integrity}}(a,t)$ &   Integrity motivation for enforcer $a$ at time $t$ \\
$u_i$ &   Utility for agent $i$ \\
$U_e(a,t)$ &   Extraction utility available to enforcer $a$ at time $t$ \\
$c$ &   Cost of cooperation \\
$b$ &   Benefit from cooperation \\
$\beta$ &   Social benefit multiplier \\
$\theta$ &   Proportion of population (typically cooperators or transformed) \\
$\theta_{\text{crit}}$ &   Critical mass threshold for stability \\
$P(t)$ &   Productive capacity at time $t$ \\
$E(t)$ &   Extraction rate at time $t$ \\
$T$ &   Time horizon \\
$p$ &   Probability (generic) \\
$p_{AI}$ &   Probability of AI-controlled TCS per cycle \\
$p_s$ &   Probability of success for voluntary coordination \\
$\lambda$ &   Average cycle duration (corruption $\to$ TCS $\to$ corruption) \\
$S$ &   Soteriological framework $(T, P, M_{\text{trans}}, \phi)$ \\
$T$ &   Telos (ultimate purpose for humans) \\
$P$ &   Set of practices for transformation \\
$\phi(S)$ &   Truth function: whether framework $S$ accurately describes reality \\
\bottomrule
\end{tabular}
\end{table}



\subsection{Conclusion}

We have established a rigorous logical chain. The trilemma establishes fundamental constraints on coordination. TCS instability eliminates technological control as viable. Trajectory inevitability shows the default path terminates in catastrophe. Game theory shows cooperation requires transformation. The resolution theorem proves VCS can work if conditions are met. Stakes analysis shows attempting VCS is rational regardless of success probability.

The mathematics proves the \emph{necessity} of voluntary coordination---it's the only option that doesn't lead to certain doom. Whether it's \emph{sufficient} (whether humanity can achieve it) remains uncertain. But when the default leads to extinction, attempting the uncertain alternative is rationally required.

The formal proofs have a profound implication that must be stated explicitly: if voluntary coordination is possible, reality has purposive structure. VCS requires a true soteriological framework with $\varphi(S) = 1$ (Theorem~\ref{thm:resolution}), which means the framework accurately describes human nature and telos. For this to be meaningful, human telos must exist objectively (not just subjectively or ``as if''). Objective human purpose means reality contains oughtness, not just is-ness. Purposive structure in reality implies something very much like intelligent design. This represents logic from survival requirements rather than a theological argument from revelation.

Consider what ``objective human purpose'' means: reality contains oughtness (``humans ought to flourish in this way''), human nature is directed toward an end, there is a right way humans should coordinate, and this rightness exists independent of opinion.

\textbf{Can purpose exist without mind?} 

Purpose means ``for the sake of X'' - it's inherently intentional. Intentionality requires intention. Intention requires mind. You cannot have purposive structure without something intelligence-like at reality's foundation.

Different traditions describe this differently (God, Tao, Logos, Brahman, Dharma), but they're pointing at the same claim: \textbf{reality has intelligence-like or mind-like properties at its foundation, not purely mechanistic/material.}

The materialist position---that there is no objective human purpose and purpose is just evolutionary selection---makes VCS impossible. If there's no objective telos, there's no true framework to discover. If $\varphi(S) = 1$ is impossible, voluntary coordination cannot resolve the trilemma. Therefore certain doom follows via the default trajectory.

The choice is binary: purposive reality (something like what religions claim) makes VCS possible and survival possible, while non-purposive reality (pure materialism) makes VCS impossible and certain doom inevitable. You cannot accept VCS works while maintaining pure materialism---the two positions are logically incompatible.

What we've proven includes weak intelligent design (reality has intelligence-like properties at its foundation), that pure materialism is false if humans can survive, that human purpose is objective and discoverable, and that atheism in the classical sense (reality is purely material) is incompatible with survival. What remains uncertain includes which specific theology is correct, whether the intelligence is personal versus impersonal, specific attributes of the foundational intelligence, and whether it's God, Brahman, Tao, Logos, or something else.

Whether you call the source of purposive structure ``God'' is somewhat semantic. The key metaphysical claim is identical across traditions: \textbf{Purpose is real, objective, and discoverable - reality has intelligence-like properties.}

We're showing that human survival requires purposive structure, and purposive structure requires something very much like intelligent design, rather than proving God through theology.

\textbf{For detailed analysis of objective ``oughtness'' and why minimal telic realism is both necessary and defensible, see4 below.}

The formal analysis provides as close to proof as we can get for claims about civilization's future. The logic is sound given the axioms. The assumptions are conservative and empirically grounded. The stakes are absolute. The metaphysical implications are unavoidable.

The choice is yours.

